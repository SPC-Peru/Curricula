\section{Situaci\'on Actual}\label{sec:cs-sit-actual}

La Facultad de Ciencias, desde hace varios a\~nos, ha contemplado esta \'area dentro de las disciplinas que fomenta, por lo que ha dedicado recursos humanos y materiales para impulsarla. Se ha incrementado su participaci\'on en el pregrado y posgrado de esta \'area. Actualmente se cuenta con un grupo de profesores dentro de la Facultad, especializados en Computaci\'on, de alto nivel acad\'emico. En los \'ultimos a\~nos este grupo ha demostrado sus capacidades, no solamente en el cumplimiento de la labor docente, sino en la impartici\'on de cursos a nivel de pregrado y posgrado, la asesor\'ia en temas de tesis, desarrollo de proyectos. En investigaci\'on se est\'an publicando trabajos en congresos nacionales y extranjeros; notas de clase y art\'iculos de divulgaci\'on e investigaci\'on.

Sin embargo, la presencia de este(os) grupo(s) dedicado(s) fuertemente a la Computaci\'on no est\'a reflejada de manera expl\'icita en la estructura de la Facultad de Ciencias y, como consecuencia, tampoco en la UNI. La Computaci\'on es una disciplina independiente con din\'amica y requerimientos propios, cuyas necesidades en ocasiones no coinciden con las necesidades de otras \'areas de conocimiento dentro de Ciencias. Por otro lado, las actividades acad\'emicas del grupo no resultan suficientemente visibles pues se pierden dentro del resto de actividades de la Facultad de Ciencias. Por estas razones, y para poder continuar con el fortalecimiento del \'area de Computaci\'on en la Universidad, se plantea la necesidad de crear la carrera de Ciencias de la Computaci\'on, dentro de la Facultad de Ciencias.

El grupo de profesores de Computaci\'on se conforma con once miembros del personal acad\'emico de tiempo completo dentro del Departamento de Matem\'aticas, que se dedican a las labores de docencia, investigaci\'on y difusi\'on de la Computaci\'on. Respecto al nivel acad\'emico del personal, forman al grupo: seis doctores, tres maestros en ciencias, y dos licenciados. M\'as a\'un, el grupo cuenta tanto con personal acad\'emico de experiencia como con elementos de reciente contrataci\'on, lo que benecia y estimula la formaci\'on acad\'emica y la investigaci\'on.

Dados los diversos intereses y habilidades del personal acad\'emico, al interior de la Facultad de Ciencias el estudio de la Computaci\'on se ha organizado en cinco \'areas: (a) C\'omputo Visual, (b) Procesamiento Paralelo y Sistemas Distribuidos, (c) Computaci\'on Cient\'ifica. Estas \'areas se definen tomando como punto de partida la clasificaci\'on tem\'atica de la disciplina realizada por la Association for Computing Machinery (ACM).\footnote{The ACM Computing Classification System [1998 Version]. CCS Up date Committee. January, 1998.}

A continuaci�n se presenta una breve descripci\'on de cada \'area desarrollada dentro del grupo, sus principales objetivos, as\'i como el personal acad\'emico que se encarga de su estudio, ense\~nanza, investigaci\'on y difusi\'on. Cabe aclarar que existen otras \'areas dentro de la Facultad de Ciencias relacionadas con la Computaci\'on, y confiamos en su participaci\'on en la futura carrera de Ciencias de la Computaci\'on.


\subsection{\'Area de C\'omputo Visual}

El C\'omputo Visual est\'a basado en la integraci\'on de tecnolog\'ias, m\'etodos y conceptos de la graficaci\'on por computadora, la percepci\'on visual, y la tecnolog\'ia y el procesamiento de im\'agenes. En las \'ultimas d\'ecadas, estas \'areas eran consideradas ajenas. Actualmente se ha visto una amplia interacci\'on entre ellas, y se considera tambi\'en su relaci\'on con la Inteligencia Artificial. El objetivo general del C\'omputo Visual es el estudio de los principios de aplicaci\'on y metodolog\'ias necesarias para la representaci\'on, manipulaci\'on y presentaci\'on de im\'agenes en dos y tres dimensiones, considerando los dispositivos de hardware y programas de software con caracter\'isticas
espec\'ificas para ello.

\subsubsection{Objetivos del \'Area}

\begin{enumerate}
\item Estudiar c\'omo se lleva a cabo el an\'alisis de la informaci\'on visual utilizando la computadora, a trav\'es del estudio de la representaci\'on y percepci\'on visual. Tanto la representaci\'on y percepci\'on visual como el an\'alisis de la informaci\'on son procesos naturales para el ser humano, pero resultan de una gran dificultad para la computadora. Sin embargo, una vez programada adecuadamente, la computadora sirve como una herramienta eficiente para la visualizaci\'on de la informaci\'on.

\item Establecer nuevos lenguajes entre el ser humano y la computadora, es decir, una manera distinta de interacci\'on: una interacci\'on visual. Para esto, la graficaci\'on por computadora permite la s\'intesis de objetos reales, de ambientes o fen\'omenos f\'isicos en objetos abstractos, basados en modelos computacionales.

\item Codificar y transformar im\'agenes, analizar escenas o reconstruir objetos en tercera dimensi\'on a partir de su proyecci\'on en dos dimensiones, mediante el procesamiento de im\'agenes.

\item Proveer de modelos gr\'aficos en ambientes simulados para la rob\'otica, para el entrenamiento de aut\'omatas con sistemas de visi\'on, sin la necesidad de escenarios reales.
\end{enumerate}

\subsubsection{Personal Acad\'emico en el \'Area}

MSc Orestes Bueno
\begin{itemize}
\item \'Areas de especializaci\'on: Matem\'aticas, Graficaci\'on 
\item Cursos: Graficaci\'on por Computadora, An\'alisis de Im\'agenes
\end{itemize}

\subsubsection{Colaboraciones Nacionales}

Centro Internacional de la Papa: Dr. Adolfo Posadas
\begin{itemize}
\item \'Areas de especializaci\'on: F\'isica, Graficaci\'on 
\item Cursos: Graficaci\'on por Computadora, An\'alisis de Im\'agenes
\end{itemize}

UPCH: Dr Mirko Zimic
\begin{itemize}
\item \'Areas de especializaci\'on: F\'isica, Graficaci\'on 
\item Cursos: An\'alisis de Im\'agenes m\'edicas
\end{itemize}

CISMID: Dr. Miguel Estrada
\begin{itemize}
\item \'Areas de especializaci\'on: Ingenier\'ia Civil, 
\end{itemize}

\subsubsection{Colaboraciones Internacionales}

IMPA - Brasil
\begin{itemize}
\item \'Areas de especializaci\'on: Graficaci\'on, Ambientes Virtuales
\end{itemize}


\subsection{\'Area de Procesamiento Paralelo y Sistemas Distribuidos}

El \'area de Procesamiento Paralelo y Sistemas Distribuidos (Ej: Metasistemas Grid) comprende el estudio de la teor\'ia, t\'ecnicas, tecnolog\'ias y m\'etodos de la fusi\'on de los dominios tradicionalmente considerados como hardware y software, haciendo \'enfasis en cada uno de sus componentes, a fin de comprender el funcionamiento de los sistemas digitales, los sistemas de c\'omputo, el dise\~no y construcci\'on de sistemas operativos, los sistemas multiprocesador y los sistemas de redes. El objetivo general es conocer la estructura y funcionamiento b\'asicos del hardware y software, formular algunas de sus especificaciones, saber c\'omo se integran equipos de c\'omputo, as\'i como las formas de distribuir y compartir recursos computacionales, procesos e informaci\'on. Con estas bases puede participar en el dise\~no de nuevas organizaciones para los sistemas de c\'omputo.

\subsubsection{Ob jetivos del \'Area}

El \'area de Computaci\'on Paralela y Metasistemas Grid tiene los siguientes ob jetivos: 
\begin{enumerate}
\item Proporcionar elementos te\'oricos y pr\'acticos para analizar y comprender los sistemas y arquitecturas de c\'omputo, as\'i como su especificaci\'on y dise\~no a lo largo de la evoluci\'on hist\'orica de las computadoras, y analizar nuevas tendencias.
\item Estudiar la teor\'ia, t\'ecnicas y metodolog\'ias para el dise\~no y construcci\'on de ensambladores, int\'erpretes y compiladores.
\item Estudiar la teor\'ia y conocer los elementos operativos requeridos para la transmisi\'on y recepci\'on de informaci\'on, as\'i como las convenciones empleadas para la comunicaci\'on entre las partes constitutivas de las redes de c\'omputo y comunicaciones.
\item Estudiar la teor\'ia, t\'ecnicas y metodolog\'ias para el dise\~no y construcci\'on de sistemas operativos, as\'i como el dise\~no e implementaci\'on de los componentes de software que hacen posible el funcionamiento de las computadoras en diferentes niveles operativos.
\item Estudiar los elementos te\'oricos, las caracter\'isticas y las propiedades de los diferentes modelos de procesamiento paralelo y distribuido, as\'i como sus comp onentes, con el fin de dise\~nar e implementar aplicaciones espec\'ificas.
\item Estudiar diversas aplicaciones que tienen los sistemas de c\'omputo avanzados.
\end{enumerate}

\subsubsection{Personal Acad\'emico en el \'Area}

Dr. Carlos Javier Solano Salinas
\begin{itemize}
\item \'Areas de especializaci\'on: F\'isica de Altas Energ\'ias, F\'isica Computacional, Metasistemas Grid
\item Cursos: Introducci\'on a Ciencias de la Computaci\'on I, Sistemas Operativos, Redes de Computadoras, Procesamiento Paralelo y Sistemas Distribuidos, Seminario de Aplicaciones de C\'omputo, Arquitectura y Dise\~no de Software
\end{itemize}

\subsubsection{Colaboraciones Nacionales}

UPCH: Dr. Jesus Castagnetto
\begin{itemize}
\item \'Areas de especializaci\'on: Qu\'imica, Bio-inform\'atica, Metasistemas Grid
\item Cursos: Sistemas Operativos, Procesamiento Paralelo y Sistemas Distribuidos
\end{itemize}

FIIS-UNI: MSc Glen Rodriguez
\begin{itemize}
\item \'Areas de especializaci\'on: Ingenier\'ia de Sistemas, Telecomunicaciones, Procesamiento Paralelo 
\item Cursos: Redes de Computadoras, Procesamiento Paralelo y  Sistemas Distribuidos.
\end{itemize}

FEE-UNI: MSc Fernando Bello
\begin{itemize}
\item \'Areas de especializaci\'on: F\'isica, Ingenier\'ia de Sistemas
\item Cursos: Arquitectura y Dise\~no de Software, Programaci\'on Orientada a Objetos.
\end{itemize}

FEE-UNI: Dr. Daniel Diaz
\begin{itemize}
\item \'Areas de especializaci\'on: Ingenier\'ia Electr\'onica
\item Cursos: Redes de Computadoras, Protocolos
\end{itemize}

\subsubsection{Colaboraciones Internacionales}

FERMILAB - EEUU
\begin{itemize}
\item  \'Areas de especializaci\'on: F\'isica de Altas Energ\'ias, F\'isica Computacional, Metasistemas Grid 
\end{itemize}

USB - Venezuela
\begin{itemize}
\item  \'Areas de especializaci\'on: Ciencias de la Computaci\'on, Metasistemas Grid 
\end{itemize}


\subsection{\'Area de Computaci\'on Cient\'ifica}

Computaci�n Cient�fica es la ciencia aplicada que se encarga de desarrollar los modelos matem�ticos y computacionales de los procesos vinculados a los problemas cient�ficos o tecnol�gicos de las ciencias naturales o de la ingenier�a. Estos modelos sirven para manipular y controlar el problema real al que representan.

Durante la formaci�n profesional se llevan b�sicamente cursos de matem�ticas, f�sica y computaci�n, complement�ndose con cursos de Biolog�a, Ecolog�a, Meteorolog�a, etc.

\subsubsection{Objetivos del \'Area}

El \'area de Computaci\'on Cient\'ifica tiene los siguientes objetivos: 
\begin{enumerate}
\item Construir modelos matem\'aticos y computacionales que se dan en las ciencias naturales e ingenier\'ia.
\item Adoptar una l\'inea de investigaci\'on en matem\'atica aplicada y computacional.
\item Brindar accesoria a las instituciones p\'ublicas y privadas dedicadas al modelaje matem\'atico, modelaje computacional y visualizaci\'on cient\'ifica.
\item Elaborar software cient\'ifico propio.
\item Abordar satisfactoriamente los problemas nacionales de car\'acter cient\'ifico o tecnol\'ogico en diversos campos de prioridad como la prospecci\'on minera, explotaci\'on petrolera, fen\'omenos marinos, problemas climatol\'ogicos y meteorol\'ogicos, etc., produciendo modelos matem�\'aticos y computacionales para su manejo, control o soluci\'on
\end{enumerate}

\subsubsection{L\'ineas del \'Area}

\begin{enumerate}
\item Bio-inform\'atica
\item F\'isica Computacional
\item Qu\'imica Computacional
\item Matem\'aticas Computacional
\end{enumerate}

\subsubsection{Personal Acad\'emico en el \'Area}

Dr. Abel Gutarra
\begin{itemize}
\item \'Areas de especializaci\'on: F\'isica, Ciencias de los Materiales
\item Cursos: Biosensores
\end{itemize}

Dr. Domingo Aliaga
\begin{itemize}
\item \'Areas de especializaci\'on: F\'isica del Estado S\'olido
\item Cursos: F\'isica Computacional
\end{itemize}

Dr. C. Javier Solano S.
\begin{itemize}
\item \'Areas de especializaci\'on: F\'isica de Altas Energ\'ias, F\'isica Computacional, Metasistemas Grid
\item Cursos: F\'isica Computacional
\end{itemize}

\subsubsection{Colaboraciones Nacionales}

UPCH: Dr. Jesus Castagnetto
\begin{itemize}
\item \'Areas de especializaci\'on: Qu\'imica, Bio-inform\'atica, Metasistemas Grid
\item Cursos: Bio-inform\'atica
\end{itemize}

UPCH: Dr. Mirko Zimic
\begin{itemize}
\item \'Areas de especializaci\'on: F\'isica, Bio-inform\'atica
\item Cursos: Bio-inform\'atica
\end{itemize}

UPCH: Dr. Jos\'e Luis Segovia
\begin{itemize}
\item \'Areas de especializaci\'on: Qu\'imica, Ciencias de la Computaci\'on
\item Cursos: Bio-inform\'atica
\end{itemize}
