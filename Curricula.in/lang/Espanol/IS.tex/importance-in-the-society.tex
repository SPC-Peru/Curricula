\section{Importancia de la carrera en la sociedad}\label{sec:importance-in-the-society}
Tomando en cosideraci�n que estamos en la era de la informaci�n, es claro que cualquier 
toma de decisiones organizacionales depende de que se disponga 
de la informaci�n adecuada en el momento en que la necesitamos.

En los tiempos en que vivimos la cantidad de informaci�n que una organizaci�n genera 
es abrumadora y ya no es posible procesar tanta informaci�n de forma manual. 
Por esa raz�n, carreras dedicadas a la administraci�n de las organizaciones 
presentan una seria limitante pues, de forma manual, s�lo les es posible analizar 
un volumen reducido de datos para tomar decisiones. 

Por otro lado, existen los profesionales altamente tecnol�gicos que crean tecnolog�a 
pero no tienen fuertes bases en temas organizacionales. El profesional en 
Sistemas de Informaci�n aparece como un excelente punto de equilibrio que sabe llevar la 
tecnolog�a a la organizaci�n para hacerla eficiente a gran escala en um ambiente globalizado.

Nuestro profesional tambi�n tiene la ventaja de poder tomar decisiones con mayor rapidez y precisi�n 
que un gerente tradicional pues �l est� preparado para procesar informaci�n voluminosa de forma directa. 
Es esta informaci�n procesada la que finalmente le da los elementos necesarios para
para una adecuada toma de decisiones organizacionales.

Finalmente, debemos tener siempre en mente que los alumnos que ingresan hoy 
saldr�n al mercado dentro de 5 a�os aproximadamente y, en un mundo que cambia tan r�pido, 
no podemos ni debemos ense�arles tomando en cuenta solamente el mercado actual y/local. 
Nuestros profesionales deben estar preparados para resolver los problemas 
que habr� dentro de 10 o 15 a�os y eso s�lo es posible a trav�s de formaci�n de la habilidad 
de aprender a aprender por su propia cuenta. Esta habilidad no s�lo es necesaria para para una larga 
vida profesional activa sino que de eso depende directamente que sea un motor de impulso en las 
organizaciones donde se desempe�e.
