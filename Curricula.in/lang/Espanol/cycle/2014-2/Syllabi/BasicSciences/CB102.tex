\begin{syllabus}

\course{CB102. An�lisis Matem�tico I}{Obligatorio}{CB102}

\begin{justification}
Un aspecto muy importante en el nivel universitario lo constituye el c�lculo diferencial,  aspecto que constituye la piedra angular de las posteriores asignaturas de matem�ticas as� como de la utilidad de la matem�tica en la soluci�n de problemas aplicados a la ciencia y la tecnolog�a. Cualquier profesional con rango universitario debe por lo tanto tener conocimiento amplio de esta asignatura, pues se convertir� en su punto de partida para los intereses de su desarrollo profesional; as� tambi�n ser� soporte para no tener dificultades en las asignaturas de matem�tica y f�sica de toda la carrera.
\end{justification}

\begin{goals}
\item Asimilar y manejar los conceptos de funci�n, sucesi�n y relacionarlos con los de l�mites y continuidad.
\item Describir, analizar, dise�ar y formular modelos continuos que dependan de una variable.
\item Conocer y manejar la propiedades del c�lculo diferencial y aplicarlas a la resoluci�n de problemas.
\end{goals}

\begin{outcomes}
\ExpandOutcome{a}{3}
\ExpandOutcome{i}{3}
\ExpandOutcome{j}{4}
\end{outcomes}

\begin{unit}{N�meros reales y funciones}{Simmons95,Bartle99}{20}{3}
   \begin{topics}
      \item N�meros reales
      \item Funciones de variable real
   \end{topics}

   \begin{unitgoals}
      \item Comprender la importancia del sistema de los n�meros reales (construcci�n), manipular los axiomas algebraicos y de orden.
      \item Comprender el concepto de funci�n. Manejar dominios, operaciones, gr�ficas, inversas.
      \end{unitgoals}
\end{unit}

\begin{unit}{Sucesiones num�ricas de n�meros reales}{Avila93,Bartle99}{18}{3}
   \begin{topics}
      \item Sucesiones
      \item Covergencia
      \item L�mites. Operaciones con sucesiones
   \end{topics}

   \begin{unitgoals}
      \item Entender el concepto de sucesi�n y su importancia.
      \item Conecer los principales tipos de sucesiones, manejar sus propiedades
      \item Manejar y calcular l�mites de sucesiones
      \end{unitgoals}
\end{unit}

\begin{unit}{L�mites de funciones y continuidad}{Apostol97,Avila93,Simmons95}{14}{4}
   \begin{topics}
      \item L�mites
      \item Continuidad
      \item Aplicaciones de funciones continuas. Teorema del valor intermedio
   \end{topics}

   \begin{unitgoals}
      \item Comprender el concepto de l�mite. calcular l�mites
      \item Analizar la continuidad de una funci�n
      \item Aplicar el teorema del valor intermedio
      \end{unitgoals}
\end{unit}

\begin{unit}{Diferenciaci�n}{Apostol97,Bartle99,Simmons95}{18}{4}
   \begin{topics}
      \item Definici�n. reglas de derivaci�n
      \item Incrementos y diferenciales
      \item Regla de la cadena. Derivaci�n impl�cita
   \end{topics}

   \begin{unitgoals}
      \item Comprender el concepto de derivada e interpretarlo.
      \item Manipular las reglas de derivaci�n
      \end{unitgoals}
\end{unit}

\begin{unit}{Aplicaciones}{Simmons95,Apostol97}{20}{4}
   \begin{topics}
      \item Funciones crecientes, decrecientes
      \item Extremos de funciones
      \item Raz�n de cambio
      \item L�mites infinitos
      \item Teorema de Taylor
   \end{topics}

   \begin{unitgoals}
      \item Utilizar la derivada para hallar extremos de funciones
      \item Resolver problemas aplicativos
      \item Utilizar el Teorema de Taylor
      \end{unitgoals}
\end{unit}



\begin{coursebibliography}
\bibfile{BasicSciences/CB102}
\end{coursebibliography}

\end{syllabus}
