\begin{syllabus}

\course{FG206. Sociolog�a}{Electivo}{FG206}

\begin{justification}
La complejidad de la vida social y la rapidez con que se suceden los cambios sociales requieren 
de explicaciones que vayan mas all� del sentido com�n, en este sentido la sociolog�a 
aporta con nuevas ideas y perspectivas de explicaci�n a los problemas que la modernidad 
ha ido generando.
\end{justification}

\begin{goals}
\item Identificar las tendencias clasicas y actuales en sociolog�a.
\end{goals}

\begin{outcomes}
\ExpandOutcome{HU}{3}
\end{outcomes}

\begin{unit}{La Sociedad}{Giddens,Gelles}{0}{2}
    \begin{topics}
      \item La sociedad
      \item Factores que identifican a las sociedades
      \item Evoluci�n de las sociedades
      \item La sociolog�a como ciencia
      \item La Relaci�n Principal Realidad Social y conocimiento de la realidad social.
    \end{topics}
    \begin{unitgoals}
      \item Discutir el concepto de sociedad y los elementos que permiten clasificarla
      \item La sociolog�a como Ciencia.
    \end{unitgoals}
\end{unit}

\begin{unit}{La Investigaci�n}{Giddens,Gelles}{0}{2}
    \begin{topics}
      \item Proceso de Investigaci�n
      \item Rol de la teor�a en la Investigaci�n
      \item La investigaci�n cuantitativa
      \item Encuesta
      \item Censo
      \item Modelos
      \item La investigaci�n cualitativa
      \item Grupo focal
      \item Observaci�n
      \item Entrevista
      \item La historia de vida
    \end{topics}
    \begin{unitgoals}
      \item Analizar los m�todos de la Investigaci�n Sociol�gica.
    \end{unitgoals}
\end{unit}

\begin{unit}{Procesos de Socializaci�n}{Giddens,Gelles}{0}{2}
    \begin{topics}
      \item La cultura una visi�n Global.
      \item La socializaci�n a trav�s del curso de la vida.
      \item Grupos y Organizaciones sociales
      \item Las estructuras sociales
    \end{topics}
    \begin{unitgoals}
      \item Analizar los procesos de socializaci�n y factores que la condicionan.
    \end{unitgoals}
\end{unit}

\begin{unit}{Organizaci�n de la Sociedad}{Giddens,Gelles}{0}{2}
    \begin{topics}
      \item Estratificaci�n social
      \item Estratificaci�n Racial
      \item Estratificaci�n de genero
    \end{topics}
    \begin{unitgoals}
      \item Analizar las formas en que se organiza y estratifica la sociedad.
    \end{unitgoals}
\end{unit}

\begin{unit}{Cambios en la Sociedad}{Giddens,Gelles}{0}{2}
    \begin{topics}
      \item Globalizaci�n
      \item Impacto en nuestras vidas
      \item Los efectos en las sociedades subdesarrolladas
      \item Las desigualdades sociales
    \end{topics}
    \begin{unitgoals}
      \item An�lisis de un mundo en cambio
    \end{unitgoals}
\end{unit}

\begin{unit}{Comunicaci�n y Tecnolog�a}{Giddens,Gelles}{0}{2}
    \begin{topics}
      \item Los medios de comunicaci�n
      \item El Impacto de la Televisi�n
      \item Las teor�as de la comunicaci�n social
      \item Nuevas tecnolog�as de comunicaci�n
      \item Globalizaci�n y medios de comunicaci�n
    \end{topics}
    \begin{unitgoals}
      \item An�lisis de un mundo en cambio
    \end{unitgoals}
\end{unit}



\begin{coursebibliography}
\bibfile{GeneralEducation/FG101}
\end{coursebibliography}

\end{syllabus}
