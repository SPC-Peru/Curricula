\begin{syllabus}

\course{CS370. T�picos en Bases de Datos}{Obligatorio}{CS370}

\begin{justification}
La gesti�n de la informaci�n (IM) juega un rol principal en casi todas las �reas donde los computadores son usados. Esta �rea incluye la captura, digitalizaci�n, representaci�n, organizaci�n, transformaci�n y presentaci�n de informaci�n; algor�tmos para mejorar la eficiencia y efectividad del acceso y actualizaci�n de informaci�n almacenada, modelamiento de datos y abstracci�n, y t�cnicas de almacenamiento de archivos f�sicos.

Este tambi�n abarca la seguridad de la informaci�n, privacidad, integridad y protecci�n en un ambiente compartido. Los estudiantes necesitan ser capaces de desarrollar modelos de datos conceptuales y f�sicos, determinar que m�todos de (IM) y t�cnicas son apropiados para un problema dado, y ser capaces de seleccionar e implementar una apropiada soluci�n de IM que refleje todas las restricciones aplicables, incluyendo escalabilidad y usabilidad.
\end{justification}

\begin{goals}
\item Llevar al alumno hacia el conocimiento de los nuevos desaf�os y complejidades de las bases de datos.
\item Hacer que el alumno cree prototipos de motores de bases de datos para la recuparaci�n de informaci�n orientada a datos complejos (imagenes, sonido, hipertexto, etc).
\end{goals}

\begin{outcomes}
\ExpandOutcome{b}{4}
\ExpandOutcome{d}{3}
\ExpandOutcome{e}{3}
\ExpandOutcome{g}{3}
\ExpandOutcome{h}{4}
\ExpandOutcome{i}{3}
\ExpandOutcome{j}{3}
\end{outcomes}

\begin{unit}{\IMDataMiningDef}{tan05,witten01,han01,kimball04,inmon04,kimball05}{10}{4}
    \IMDataMiningAllTopics%
    \IMDataMiningAllObjectives%
\end{unit}

\begin{unit}{\IMHypermidiaDef}{brusilovsky98,elmasri04}{10}{4}
    \IMHypermidiaAllTopics%
    \IMHypermidiaAllObjectives%
\end{unit}

\begin{unit}{\IMMultimediaSystemsDef}{elmasri04}{10}{4}
    \IMMultimediaSystemsAllTopics%
    \IMMultimediaSystemsAllObjectives%
\end{unit}

\begin{unit}{\IMDigitalLibrariesDef}{witten02,elmasri04}{10}{4}
    \IMDigitalLibrariesAllTopics%
    \IMDigitalLibrariesAllObjectives%
\end{unit}



\begin{coursebibliography}
\bibfile{Computing/CS/CS270W}
\end{coursebibliography}

\end{syllabus}
