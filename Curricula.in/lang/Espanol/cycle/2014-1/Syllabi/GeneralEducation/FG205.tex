\begin{syllabus}

\course{FG205. Historia de la Cultura}{Obligatorio}{FG205}

\begin{justification}
Asignatura b�sica de car�cter formativo y human�stico. Resulta fundamental encontrar justificados o naturales los actos y los principales hitos de la historia universal desde una perspectiva cultural que procure ser profunda, estructurada y cr�tica. Este conocimiento permitir� entender mejor el presente para proyectarnos con sabidur�a al futuro.
\end{justification}

\begin{goals}
\item Comprender que la formaci�n de un buen profesional no se desliga ni se opone, m�s bien contribuye al aut�ntico crecimiento personal. Esto requiere de la asimilaci�n de valores s�lidos, horizontes culturales amplios y una visi�n profunda del entorno cultural.
\end{goals}

\begin{outcomes}
\ExpandOutcome{FH}{2}
\ExpandOutcome{HU}{3}
\end{outcomes}

\begin{unit}{}{hubenak2006historia}{6}{2}
\begin{topics}
	\item Los motivos del estudio de la historia. 	
	\item Historia como ciencia. 	
	\item �Qu� es Occidente? 	
	\item Cultura. 	
	\item El cristianismo y la cultura. 
\end{topics}
\begin{unitgoals}
	\item Conocer nociones te�ricas sobre la concepci�n, posibilidades y l�mites de la Historia de la Cultura y obtener nociones b�sicas de la historia de la cultura universal.
\end{unitgoals}
\end{unit}

\begin{unit}{}{hubenak2006historia}{9}{2}
\begin{topics}
	\item El Mundo Hel�nico. 	
	\item El mundo Romano. 	
	\item Herencia cultural greco-romana. 
\end{topics}
\begin{unitgoals}
	\item Conocer las bases greco-latinas de la cultura occidental.
\end{unitgoals}
\end{unit}

\begin{unit}{}{hubenak2006historia}{12}{2}
\begin{topics}
	\item La Romanidad y la Iglesia: pilares b�sicos de la civilizaci�n occidental. 	
	\item Surgimiento y desarrollo de la edad Media.
\end{topics}
\begin{unitgoals}
	\item Comprender la transformaci�n del mundo romano en cristiano, su preservaci�n y apogeo.
\end{unitgoals}
\end{unit}

\begin{unit}{}{hubenak2006historia}{9}{2}
\begin{topics}
	\item El renacimiento y el nacimiento de la imagen moderna del mundo.  	
	\item La �poca de las revueltas. 	
	\item La Reforma Cat�lica. 	
	\item La ilustraci�n y el endiosamiento de la raz�n. 
\end{topics}
\begin{unitgoals}
	\item Percibir la desintegraci�n de la unidad y el ideal cristiano.
\end{unitgoals}
\end{unit}

\begin{unit}{}{hubenak2006historia}{9}{2}
\begin{topics}
	\item Revoluciones burguesas en Europa e independencia de pa�ses latinoamericanos en el continente americano. 	
	\item El Mundo Contempor�neo y el modernismo. 	
	\item El comienzo de la crisis del siglo XX: guerra y revoluci�n. 	
	\item La infructuosa b�squeda de una nueva estabilidad: Europa entre las guerras 1919-1939. 	
	\item La profundidad de la crisis europea: la Segunda Guerra Mundial. 	
	\item La Guerra Fr�a y la nueva Europa. Albores del siglo XXI.
\end{topics}
\begin{unitgoals}
	\item Aprehender el desarrollo final de la crisis del mundo occidental con la consecuente crisis de los valores cristianos: el materialismo, el hedonismo, el relativismo en la pr�ctica de la vida de las sociedades. 
\end{unitgoals}
\end{unit}

\begin{unit}{La Ilustraci�n}{hubenak2006historia}{5}{2}
\begin{topics}
	\item La ilustraci�n y el endiosamiento de la raz�n.
\end{topics}
\begin{unitgoals}
	\item Valorar las nuevas ideas desarrolladas por este movimiento cultural y sus repercusiones.
\end{unitgoals}
\end{unit}



\begin{coursebibliography}
\bibfile{GeneralEducation/FG101}
\end{coursebibliography}

\end{syllabus}
