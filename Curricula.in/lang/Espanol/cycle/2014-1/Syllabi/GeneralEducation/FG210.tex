\begin{syllabus}

\course{FG210. �tica}{Obligatorio}{FG210}

\begin{justification}
Brindar al alumno criterios de discernimiento general y particular, as� como pautas morales para que con ellos oriente su conducta personal, de modo que se oriente a su realizaci�n integral mediante actos queridos, conscientes, libres y responsables. 
\end{justification}

\begin{goals}
\item Formar la conciencia del estudiante para que pueda conducirse moralmente en el �mbito personal y profesional.
\end{goals}

\begin{outcomes}
\ExpandOutcome{e}{2}
\ExpandOutcome{FH}{2}
\ExpandOutcome{TASDSH}{2}
\end{outcomes}

\begin{unit}{La �tica Filos�fica}{Lewis, Bourmaud, RodriguezL, AristotelesE}{9}{2}
\begin{topics}
	\item	Presentaci�n del curso. 
	\item	Lo �tico y moral. La �tica como rama de la filosof�a.
	\item	La necesidad de la metaf�sica.
	\item	La experiencia moral.
	\item	El problema del relativismo y su soluci�n.
	
\end{topics}
\begin{unitgoals}
	\item Presentar una primera noci�n de la �tica y de los problemas relativos a esta rama de la filosof�a.
\end{unitgoals}
\end{unit}

\begin{unit}{La acci�n moral}{SanchezM,Genta}{15}{4}
\begin{topics}
	\item	Caracterizaci�n del actuar humano. 
	\item	Libertad, conciencia y voluntariedad. Distintos niveles de libertad. Factores que afectan la voluntariedad.
	\item	El papel de la afectividad en la moralidad.
	\item	La felicidad como fin �ltimo del ser humano.

\end{topics}
\begin{unitgoals}
	\item Hacer un an�lisis del acto humano, presentando sus condiciones y especificando su moralidad.
\end{unitgoals}
\end{unit}

\begin{unit}{La vida virtuosa}{Piper,Droste,Lego,StoTomas}{12}{4}
\begin{topics}
	\item	Qu� se entiende por virtud.
	\item	La virtud moral: caracterizaci�n y modo de adquisici�n; el car�cter din�mico de la virtud.
	\item	Relaci�n entre las distintas virtudes �ticas. Las virtudes cardinales. Los vicios.

\end{topics}
\begin{unitgoals}
	\item Presentar el ideal filos�fico de la vida virtuosa destacando algunas virtudes fundamentales.
\end{unitgoals}
\end{unit}

\begin{unit}{Lo �ticamente correcto y su conocimiento}{ReydeCastro2010,SanchezM,Genta}{9}{4}
\begin{topics}
	\item 	La correcci�n en lo �tico.
	\item 	El conocimiento de lo �ticamente correcto.
	\item 	La llamada ``recta raz�n'' y la ``verdad pr�ctica''. 
	\item 	Las leyes morales: ley natural y ley positiva.
	\item 	La conciencia moral: definici�n, tipos, deformaciones. 
	\item 	La valoraci�n moral de las acciones concretas.

\end{topics}

\begin{unitgoals}
	\item Presentar las nociones de recta raz�n, conciencia moral, y moral natural destacando el conocimiento de la ley moral natural.
\end{unitgoals}
\end{unit}



\begin{coursebibliography}
\bibfile{GeneralEducation/FG101}
\end{coursebibliography}

\end{syllabus}
