\begin{syllabus}

\course{FG350. Liderazgo y Desempe�o}{Obligatorio}{FG350}

\begin{justification}
El mundo de hoy y las organizaciones existentes exigen de l�deres que permitan orientarlas hacia la construcci�n de una sociedad m�s justa y reconciliada.  Ese desaf�o pasa por la necesidad de formar personas con un recto conocimiento de s� mismos, con la capacidad de juzgar objetivamente la realidad y de proponer orientaciones que busquen modificar positivamente el entorno.

El curso de Liderazgo y Desempe�o pretende desarrollar los criterios, habilidades y actitudes necesarios para cumplir con �ste prop�sito.
\end{justification}

\begin{goals}
\item \OutcomeHU
\item Mostrar la influencia del liderazgo a trav�s de la historia.
\item Dar a conocer la imortancia de un liderazgo equilibrado en nuestra sociedad.
\item Forjar en el alumno un desempe�o honesto y preciso.
\end{goals}

\begin{outcomes}
\ExpandOutcome{d}{3}
\ExpandOutcome{f}{3}
\ExpandOutcome{HU}{3}
\end{outcomes}

\begin{unit}{Aproximaci�n al liderazgo}{NotasLiderazgo2006,TheodorHaecker,Guardini1992,Hesselbein1999}{20}{2}
\begin{topics}
	\item Introducci�n al liderazgo
	\item Estilos actuales de liderazgo
	\item Visiones erradas del ser humano
	\item La vocaci�n humana
	\item Ensayando una definici�n de liderazgo
	\item Liderazgo en la historia
	\item Importancia de las aproximaciones hist�ricas
	\item Elementos para analizar un liderazgo hist�rico
\end{topics}
\begin{unitgoals}
	\item Conocer las caracter�sticas del liderazgo, su importancia y trascendencia a trav�s de la historia.
\end{unitgoals}
\end{unit}

\begin{unit}{Liderazgo personal/Maestr�a personal}{NotasLiderazgo2006,TheodorHaecker,Guardini1992,Hesselbein1999}{45}{3}
\begin{topics}
	\item Introducci�n al liderazgo personal
	\item El primer campo de liderazgo soy yo
	\item Autoridad y liderazgo
	\item Introducci�n al autoconocimiento y liderazgo
	\item El ruido
	\item Hacer silencio
	\item Obst�culos para el autoconocimiento
	\item Empezando a conocerme
	\item Que no es conocerme
	\item Aproximaci�n al autoconocimiento.
	\item El hombre unidad de mente cuerpo y esp�ritu.
	\item El cuerpo
	\item La mente
	\item El esp�ritu
	\item Caracter�sticas de la mismidad
	\item La libertad
	\item La dimisi�n de lo humano
	\item La Prudencia
	\item Toma de conciencia
	\item Mi liderazgo personal
	\item An�lisis FODA personal
	\item Plan de vida
	\item Manejo de horario
\end{topics}
\begin{unitgoals}
	\item Entender que el primer campo de liderazgo es la misma persona
	\item Profundizar en el descubrimiento del misterio de la persona humana
	\item Desarrollar habilidades y actitudes de l�der
\end{unitgoals}
\end{unit}

\begin{unit}{Liderazgo en grupos}{NotasLiderazgo2006,TheodorHaecker,Guardini1992,Hesselbein1999}{10}{3}
\begin{topics}
	\item La relaci�n personal con el equipo
	\item Liderazgo integral
	\item Acompa�amiento y discipulado
	\item Fundamentos de unidad
\end{topics}
\begin{unitgoals}
	\item Desarrollar habilidades para el trabajo en equipo
\end{unitgoals}
\end{unit}



\begin{coursebibliography}
\bibfile{GeneralEducation/FG350}
\end{coursebibliography}

\end{syllabus}
