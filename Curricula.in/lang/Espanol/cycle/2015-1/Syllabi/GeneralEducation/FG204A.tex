\begin{syllabus}

\course{FG204A. Teolog�a II}{Obligatorio}{FG204A}

\begin{justification}
La Universidad Cat�lica San Pablo busca ofrecer una visi�n de la persona humana y del mundo iluminada por el Evangelio y, consiguientemente, por la fe en Cristo-Logos, como centro de la creaci�n y de la historia. El estudio de la teolog�a es fundamental para dicha comprensi�n de Dios, del hombre y del cosmos.
La Teolog�a permite al creyente en Cristo conocer y comprender mejor su fe. Al no creyente, la comprensi�n de la cosmovisi�n que ha forjado la cultura occidental en la cual ha nacido, vive y desarrollar� su propia vida, as� como abrirse al conocimiento de Dios desde Jesucristo y su Iglesia.
El curso de Teolog�a II le permitir� al alumno  adentrarse en la comprensi�n de los contenidos y consecuencias del Dogma cristiano.
\end{justification}

\begin{goals}
\item Conocer y comprender el Cristianismo en cuanto religi�n revelada desde las razones en las que se apoya, mostrando su credibilidad, a fin de ofrecer al creyente razones que motivan su opci�n de fe y presentar a quien no lo es razones para creer. [\Usage]
\end{goals}

\begin{outcomes}
    \item \ShowOutcome{n}{1}
    \item \ShowOutcome{�}{3}
\end{outcomes}

\begin{competences}
    \item \ShowCompetence{C22}{n} 
    \item \ShowCompetence{C24}{�}
\end{competences}

\begin{unit}{}{Dios en s�}{pablo1998creo,pozo1968,ratzinger2005,Ibanez2005}{9}{C22,C24}
\begin{topics}
	\item Dios Uno y Trino.
	    \begin{subtopics}
		\item Dios plenitud del Ser.
		\item Los atributos divinos.
		\item Dios es plenitud del Amor.
		\item Dios Padre, Hijo y Esp�ritu Santo, Comuni�n de Amor.
	    \end{subtopics}
\end{topics}
\begin{learningoutcomes}
	\item Que el  alumno reflexione sobre el problema de Dios para la humanidad y los atributos del Ser Divino. [\Familiarity]
\end{learningoutcomes}
\end{unit}

\begin{unit}{}{Dios creador}{pablo1998creo,Catecismo,guardini2006}{6}{C22,C24}
\begin{topics}
	\item La pregunta sobre la creaci�n.
	      \begin{subtopics}
		\item La revelaci�n de la creaci�n como obra de Dios.
		\item Creaci�n obra de la Trinidad.
		\item Caracter�sticas.
		\item Errores acerca de la creaci�n.
		\item El hombre Se�or de la Creaci�n.
		\item La Providencia.
	      \end{subtopics}
\end{topics}
\begin{learningoutcomes}
	\item Que el  alumno reflexione sobre Dios Creador como fundamento de toda la realidad.[\Familiarity]
\end{learningoutcomes}
\end{unit}

\begin{unit}{}{El pecado y la reconciliaci�n en Cristo}{Ratzinger2007,Catecismo,DelaPotterie1997,ratzinger2005}{12}{C22,C24}
\begin{topics}
	\item El Problema del Mal.
	    \begin{subtopics}
		\item El Demonio.
		\item El Pecado Original Originante.
		\item El Pecado Original Originado.
	    \end{subtopics}
	\item Cristo Reconciliador
	      \begin{subtopics}
		\item T�rminos Soteriol�gicos.
		\item Ciclo reconciliador.
		\item Reconciliaci�n: Nueva Creaci�n.
	      \end{subtopics}
\end{topics}
\begin{learningoutcomes}
	\item Que el alumno reflexione sobre el contenido de la fe en Jesucristo que parte de lo que �l dijo de s� mismo y es recogido por sus primeros testigos. [\Familiarity]
\end{learningoutcomes}
\end{unit}

\begin{unit}{}{La gracia y las realidades �ltimas}{Collantes2011,Catecismo,Aquino1943}{18}{C22,C24}
\begin{topics}
	\item La Gracia .
	      \begin{subtopics}
		\item Santidad.
		\item Naturaleza de la Gracia.
		\item La acci�n de la Gracia.
		\item Los sacramentos.
		\item La oraci�n.
		\item La liturgia.
	      \end{subtopics}
	\item Mar�a en el misterio de Cristo y de la Iglesia.
	      \begin{subtopics}
		\item Misi�n de la Sant�sima Virgen en la econom�a de la salvaci�n.
		\item La Sant�sima Virgen y la Iglesia.
	      \end{subtopics}
	\item La Vida eterna.
	      \begin{subtopics}
		\item Escatolog�a intermedia.
		\item La Resurrecci�n de la Carne.
		\item El infierno.
		\item La Vida Eterna.
	      \end{subtopics}
\end{topics}
\begin{learningoutcomes}
	\item Que el alumno reflexione sobre la gracia y las realidades �ltimas de la fe. [\Usage]
\end{learningoutcomes}
\end{unit}



\begin{coursebibliography}
\bibfile{GeneralEducation/FG204}
\end{coursebibliography}
\end{syllabus}
