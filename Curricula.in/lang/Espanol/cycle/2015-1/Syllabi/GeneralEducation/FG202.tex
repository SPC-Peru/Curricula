\begin{syllabus}

\course{FG202. Apreciaci�n Literaria}{Electivos}{FG202}

\begin{justification}
La Universidad Cat�lica San Pablo dentro de su proyecto Educativo se�ala la importancia de la Formaci�n Humana de sus alumnos, que mejor veh�culo para contribuir a este objetivo que la Literatura que es un importante medio de expresi�n humana, a trav�s de estas  conocemos el ama de los pueblos y el pensamiento, vivencias, sue�os, sufrimientos y esperanzas del hombre a trav�s de los tiempos.
\end{justification}

\begin{goals}
\item Este curso contribuye a entender la literatura como un medio de expresi�n del ser humano.
\end{goals}

\begin{outcomes}
    \item \ShowOutcome{�}{2}
\end{outcomes}
\begin{competences}
    \item \ShowCompetence{C24}{�}
\end{competences}

\begin{unit}{}{La literatura}{Jackson,Riquer}{0}{C24}
\begin{topics}
	\item La comunicaci�n Literaria
	\item G�neros Literarios
	\item El comentario y an�lisis de testos.
	\item El lenguaje, herramienta fundamental de la Literatura: Sus posibilidades e imposibilidades.
\end{topics}
\begin{learningoutcomes}
	\item Definir y caracterizar la Literatura, indicando sus g�neros, recursos y lenguaje [\Usage].
	\item Valorar la expresi�n Literaria en su esencia, adoptando una postura de apertura y sensibilidad hacia ella [\Usage].
\end{learningoutcomes}
\end{unit}

\begin{unit}{}{El Clasicismo}{Jackson,Riquer}{3}{C24}
\begin{topics}
	\item Homero ``La Iliada''
	\item S�focles ``Edipo Rey''
	\item Virgilio ``La Eneida''
	\item Literatura Cristiana ``La Biblia''
\end{topics}
\begin{learningoutcomes}
	\item Conocer y Valorar el Clasicismo y la Literatura Cl�sica [\Usage].
	\item Leer, comentar y apreciar fragmentos selectos de Literatura Cl�sica [\Usage].
\end{learningoutcomes}
\end{unit}

\begin{unit}{}{Literatura Medieval}{Jackson,Riquer}{3}{C24}
\begin{topics}
	\item Dante Alighieri ``La Divina Comedia'' ``Poema del Mio Cid''
	\item    San Agust�n ``Confesiones''  
	\item    El Renacimiento
	\item    Shakespeare ``Hamlet''
	\item    Miguel de Cervantes ``Don Quijote''
\end{topics}
\begin{learningoutcomes}
	\item Se�alar caracter�sticas b�sicas de la Edad Media y de la Literatura Medieval[\Usage].
	\item Leer, analizar y valorar textos de literatura medieval [\Usage].
	\item Caracterizar el Renacimiento [\Usage].
	\item Valorar textos renacentistas [\Usage].
\end{learningoutcomes}
\end{unit}

\begin{unit}{}{El Romanticismo}{Jackson,Riquer}{3}{C24}
\begin{topics}
	\item Goethe ``Werther''
	\item Allan Poe ``Narraciones Extraordinarias''
	\item A. B�cquer ``Rimas y Leyendas''
	\item Mariano Melgar ``Yarav�''
	\item Realismo Ruso Fedor Dostoievski 
	\item ``Crimen y Castigo''
	\item Manuel Gonz�les Prada ``Paginas Libres''
\end{topics}
\begin{learningoutcomes}
	\item Valorar la literatura del Romanticismo a trav�s de la lectura de textos rom�nticos [\Usage].
	\item Caracterizar el Realismo y leer y analizar fragmentos de literatura realista [\Usage].
\end{learningoutcomes}
\end{unit}

\begin{unit}{}{El Modernismo}{Jackson,Riquer}{3}{C24}
\begin{topics}
	\item Rub�n Dar�o
	\item Antonio Machado
	\item El Postmodernismo-.Gabriela Mistral, C�sar Vallejo.
\end{topics}
\begin{learningoutcomes}
	\item Caracterizar e Movimiento Literario Modernista y Post-Modernista [\Usage].
	\item Leer y valorar selectos textos Modernistas y Post Modernistas [\Usage].
\end{learningoutcomes}
\end{unit}

\begin{unit}{}{Narrativa del Siglo XX}{Jackson,Riquer}{3}{C24}
\begin{topics}
	\item Franz Kafka ``Metamorfosis''
	\item Albert Camus ``La Peste''
	\item Dramatica del siglo XX
	\item Bertolt Brecht, Garcia Lorca
	\item Poesia del siglo XX
	\item Pablo Neruda
	\item Octavio Paz
	\item Javier Heraud
\end{topics}
\begin{learningoutcomes}
	\item Leer, analizar y valorar selectos textos de literatura del siglo XX, ubic�ndolos en su respectivo contexto hist�rico [\Usage].
\end{learningoutcomes}
\end{unit}

\begin{unit}{}{Lectura}{Jackson,Riquer}{3}{C24}
\begin{topics}
	\item Gabriel Garc�a M�rquez ``Cien a�os de Soledad''
	\item Arturo P�rez Reverte
	\item Camili Jos� Cela
	\item Mario Vargas Llosa, Alfredo Bryce E.
	\item Otros Autores de hoy
	\item Isabel Allende, Jos� Saramago, Paulo Coelho.
\end{topics}
\begin{learningoutcomes}
	\item Leer, analizar y valorar selectos textos de literatura [\Usage].
\end{learningoutcomes}
\end{unit}



\begin{coursebibliography}
\bibfile{GeneralEducation/FG101}
\end{coursebibliography}
\end{syllabus}
