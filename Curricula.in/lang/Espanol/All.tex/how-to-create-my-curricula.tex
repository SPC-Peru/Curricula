\section*{?`C�mo crear una malla para mi carrera?}
\addcontentsline{toc}{section}{?`C�mo crear una malla para mi carrera?}%

A pesar de que esta malla trata de abarcar la gran mayor�a del cuerpo del conocimiento de esta �rea,  
es necesario personalizarla para cada instituci�n. Para que esto sea posible, se ha dejado un 
margen de cr�ditos en cada semestre de la malla presentada en la secci�n \ref{sec:tables-by-semester}.

Hay universidades, como la \acf{UCSP} que tienen un especial �nfasis en la formaci�n de la persona. 
Estos cr�ditos libres han permitido completar el documento de manera r�pida a la carrera. 
En el caso de la \acf{UNSA} los cr�ditos libres fueron utilizados para incorporar cursos de ingl�s. 

Sabemos que cada universidad tiene un especial �nfasis en alguna l�nea en especial pero nunca 
debemos perder el foco principal de la carrera. No debemos caer nuevamente el error de crear 
decenas mallas curriculares cada una de ellas tratando de abarcar el todo de los cinco perfiles 
internacionales de IEEE-CS, ACM. En la pr�ctica lo que hemos provocado es que no sea posible 
encontrar dos carreras con el mismo nombre que sean realmente equivalentes. 

Si Ud. desea crear una malla curricular en base a este documento debe seguir los siguientes pasos:

\begin{itemize}
\item Leer el documento y entender su estructura,
\item Prestar especial atenci�n a los cursos de la malla que est�n en secci�n \ref{sec:tables-by-semester} 
      as� como a los cr�ditos libres en cada semestre.
\item Pensar en los cursos necesarios para adaptar y/o completar esta propuesta a su instituci�n,
\item Tener el logo de la instituci�n en tama�o 7x7cm en formato eps (\textit{Encrypted Poscript}) y 
      el mismo logo con transparencia de 50\% para el fondo del documento,
\item Tener algunos datos generales: nombre de la facultad, nombre de la carrera, 
\item Entrar en contacto con los autores de esta propuesta para la generaci�n del documento.
\end{itemize}
