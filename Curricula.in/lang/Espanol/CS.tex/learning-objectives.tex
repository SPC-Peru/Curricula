\section{Objetivos de aprendizaje}

En esta malla curricular Utilizamos tres niveles de dominio, los mismos que se definen como: 
\begin{description}
 \item [Familiaridad:] El estudiante entiende lo que es un concepto o lo que significa. Este nivel de dominio se refiere a un conocimiento b�sico de un concepto en lugar de esperar instalaci�n real con su aplicaci�n. Proporciona una respuesta a la pregunta "�Qu� sabe usted acerca de esto?" 
 \item [Uso:] El alumno es capaz de utilizar o aplicar un concepto de una manera concreta. El uso de un concepto puede incluir, por ejemplo, apropiadamente usando un concepto espec�fico en un programa, utilizando una t�cnica de prueba en particular, o la realizaci�n de un an�lisis particular. Proporciona una respuesta a la pregunta "�Qu� sabes c�mo hacerlo?" 
 \item [Evaluaci�n]: El alumno es capaz de considerar un concepto de m�ltiples puntos de vista y / o justificar la selecci�n de un determinado enfoque para resolver un problema. Este nivel de dominio implica m�s que el uso de un concepto; se trata de la posibilidad de seleccionar un enfoque adecuado de las alternativas conocidas. Proporciona una respuesta a la pregunta "�Por qu� hiciste eso?"
\end{description}
